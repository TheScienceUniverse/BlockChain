\chapter{Details}
\label{ch:3}
%\setcounter{page}{1}
%\pagenumbering{arabic}
\bigskip

The process underlying the block-chain technology are the folowing:
\begin{itemize}
\item Creating Account and Be a Part of the Network with an hash id
\item Creating own Data block
\item Request the BCT System to add it by sending it to the open Network in secure or enCrypted way
\item The system then send the block to other nodes for verification
\item The other nodes can verify it with solving some nonce and report to system
\item After being verified the System will add the new block to the chain and update it to every peer's copy of the hyper-ledger
\item Data mining is useful when to check again and again for the integrity of data and the performance of the system
\end{itemize}

The following is the overview of the BitCoin implemention using concepts of blockchain,

\section{Transaction}
We define a block data masked with chain of digital signatures. Each owner transfers the coin to the
next by digitally signing a hash of the previous transaction and the public key of the next owner
and adding these to the end of the coin. A payee can verify the signatures to verify the chain of
ownership.

\section{Timestamp}
The solution we propose begins with a timestamp server. A timestamp server works by taking a
hash of a block of items to be timestamped and widely publishing the hash, such as in a
newspaper or Usenet post [2-5]. The timestamp proves that the data must have existed at the
time, obviously, in order to get into the hash. Each timestamp includes the previous timestamp in
its hash, forming a chain, with each additional timestamp reinforcing the ones before it.

\section{Proof of Work}
To implement a distributed timestamp server on a peer-to-peer basis, we will need to use a proof-
of-work system. The proof-of-work involves scanning for a value that when hashed, such as with SHA-256, the
hash begins with a number of zero bits. The average work required is exponential in the number
of zero bits required and can be verified by executing a single hash.
For our timestamp network, we implement the proof-of-work by incrementing a nonce in the
block until a value is found that gives the block's hash the required zero bits. Once the CPU
effort has been expended to make it satisfy the proof-of-work, the block cannot be changed
without redoing the work. As later blocks are chained after it, the work to change the block
would include redoing all the blocks after it.

\section{Network}
The steps to run the network are as follows:
\begin{enumerate}
\item New transactions are broadcast to all nodes.
\item Each node collects new transactions into a block.
\item Each node works on finding a difficult proof-of-work for its block.
\item When a node finds a proof-of-work, it broadcasts the block to all nodes.
\item Nodes accept the block only if all transactions in it are valid and not already spent.
\item Nodes express their acceptance of the block by working on creating the next block in the chain, using the hash of the accepted block as the previous hash.
\end{enumerate}

\section{Incentive}
By convention, the first transaction in a block is a special transaction (the block is called genesis block) that starts a new block owned
by the creator of the block. This adds an incentive for nodes to support the network, and provides
a way to initially distribute coins into circulation, since there is no central authority to issue them.
The steady addition of a constant of amount of new coins is analogous to gold miners expending
resources to add gold to circulation. In our case, it is CPU time and electricity that is expended.

\section{Claiming Memory}
Once the latest transaction in a coin is buried under enough blocks, the spent transactions before
it can be discarded to save disk space. To facilitate this without breaking the block's hash,
transactions are hashed in a Merkle Tree, with only the root included in the block's hash.
Old blocks can then be compacted by stubbing off branches of the tree. The interior hashes do
not need to be stored.

\section{Simplified Video Verification}
It is possible to verify video without running a full network node. A user only needs to keep
a copy of the block headers of the longest proof-of-work chain, which he/she can get by querying
network nodes until he's/she's convinced he/she has the longest chain, and obtain the Merkle branch
linking the transaction to the block it's timestamped in. He can't check the transaction for
himself, but by linking it to a place in the chain, he can see that a network node has accepted it,
and blocks added after it further confirm the network has accepted it.

\section{Privacy}
The traditional banking model achieves a level of privacy by limiting access to information to the
parties involved and the trusted third party. The necessity to announce all transactions publicly
precludes this method, but privacy can still be maintained by breaking the flow of information in
another place: by keeping public keys anonymous. The public can see that someone is sending
an amount to someone else, but without information linking the transaction to anyone. This is
similar to the level of information released by stock exchanges, where the time and size of
individual trades, the "tape", is made public, but without telling who the parties were.
