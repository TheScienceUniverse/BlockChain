% Machine Learning
A machine is said to be learning from past Experiences(data feed in) with respect to some class of Tasks, if it's Performance in a given Task improves with the Experience.

According to a famous person, Machine Learning is a system that can learn from example through self-improvement and without being explicitly coded by programmer. The breakthrough comes with the idea that a machine can singularly learn from the data (\textit{i.e.} media database of real life objects, numbers dataset from algorithms, face images, job dtabase) to produce accurate results.

Machine learning combines data with statistical tools to predict an output. This output is then used by corporate to makes actionable insights. Machine learning is closely related to data mining and Bayesian predictive modeling. The machine receives data as input, use an algorithm to formulate answers.

A typical machine learning tasks are to provide a recommendation. For those who have a Netflix account, all recommendations of movies or series are based on the user's historical data. Tech companies are using unsupervised learning to improve the user experience with personalizing recommendation.

\subsection{Machine Learning vs. Traditional Programming}
Traditional programming differs significantly from machine learning. In traditional programming, a programmer code all the rules in consultation with an expert in the industry for which software is being developed. Each rule is based on a logical foundation; the machine will execute an output following the logical statement. When the system grows complex, more rules need to be written. It can quickly become unsustainable to maintain.

Machine learning is supposed to overcome this issue. The machine learns how the input and output data are correlated and it writes a rule. The programmers do not need to write new rules each time there is new data. The algorithms adapt in response to new data and experiences to improve efficacy over time.

\subsection{How does Machine learning work}
Machine learning is the virtual brain where all the learning takes place. The way the machine learns is similar to the human being. Humans learn from experience. The more we know, the more easily we can predict. By analogy, when we face an unknown situation, the likelihood of success is lower than the known situation. Machines are trained the same. To make an accurate prediction, the machine sees an example. When we give the machine a similar example, it can figure out the outcome. However, like a human, if its feed a previously unseen example, the machine has difficulties to predict.

The core objective of machine learning is the learning and inference. First of all, the machine learns through the discovery of patterns. This discovery is made thanks to the data. One crucial part of the data scientist is to choose carefully which data to provide to the machine. The list of attributes used to solve a problem is called a feature vector. You can think of a feature vector as a subset of data that is used to tackle a problem.

The machine uses some fancy algorithms to simplify the reality and transform this discovery into a model. Therefore, the learning stage is used to describe the data and summarize it into a model.

For instance, the machine is trying to understand the relationship between the wage of an individual and the likelihood to go to a fancy restaurant. It turns out the machine finds a positive relationship between wage and going to a high-end restaurant: This is the model

\subsubsection{Inferring}
When the model is built, it is possible to test how powerful it is on never-seen-before data. The new data are transformed into a features vector, go through the model and give a prediction. This is all the beautiful part of machine learning. There is no need to update the rules or train again the model. You can use the model previously trained to make inference on new data.

The life of Machine Learning programs is straightforward and can be summarized in the following points:
\begin{enumerate}
\item Define a question
\item Collect data
\item Visualize data
\item Train algorithm
\item Test the Algorithm
\item Collect feedback
\item Refine the algorithm
\item Loop 4-7 until the results are satisfying
\item Use the model to make a prediction
\end{enumerate}

There are two main types of Learning,
\subsubsection{Supervised learning}
An algorithm uses training data and feedback from humans to learn the relationship of given inputs to a given output. For instance, a practitioner can use marketing expense and weather forecast as input data to predict the sales of cans.

You can use supervised learning when the output data is known. The algorithm will predict new data.

There are two categories of supervised learning:
\paragraph{Classification}
Imagine you want to predict the gender of a customer for a commercial. You will start gathering data on the height, weight, job, salary, purchasing basket, etc. from your customer database. You know the gender of each of your customer, it can only be male or female. The objective of the classifier will be to assign a probability of being a male or a female (\textit{i.e.}, the label) based on the information (\textit{i.e.}, features you have collected). When the model learned how to recognize male or female, you can use new data to make a prediction. For instance, you just got new information from an unknown customer, and you want to know if it is a male or female. If the classifier predicts male = 70\%, it means the algorithm is sure at 70\% that this customer is a male, and 30\% it is a female.

The label can be of two or more classes. The above example has only two classes, but if a classifier needs to predict object, it has dozens of classes (\textit{e.g.} glass, table, shoes, etc. each object represents a class).

\paragraph{Regression}
When the output is a continuous value, the task is a regression. For instance, a financial analyst may need to forecast the value of a stock based on a range of feature like equity, previous stock performances, macroeconomics index. The system will be trained to estimate the price of the stocks with the lowest possible error.

\subsubsection{Unsupervised learning}
In unsupervised learning, an algorithm explores input data without being given an explicit output variable (\textit{e.g.} explores customer demographic data to identify patterns)

You can use it when you do not know how to classify the data, and you want the algorithm to find patterns and classify the data for you

Challenges and Limitations of Machine learning
The primary challenge of machine learning is the lack of data or the diversity in the dataset. A machine cannot learn if there is no data available. Besides, a dataset with a lack of diversity gives the machine a hard time. A machine needs to have heterogeneity to learn meaningful insight. It is rare that an algorithm can extract information when there are no or few variations. It is recommended to have at least 20 observations per group to help the machine learn. This constraint leads to poor evaluation and prediction.

\subsection{Application}
There are plenty of application fields where ML is being used for better performance,
\subsubsection{Augmentation}
Machine learning, which assists humans with their day-to-day tasks, personally or commercially without having complete control of the output. Such machine learning is used in different ways such as Virtual Assistant, Data analysis, software solutions. The primary user is to reduce errors due to human bias.
\subsubsection{Automation}
Machine learning, which works entirely autonomously in any field without the need for any human intervention. For example, robots performing the essential process steps in manufacturing plants.
\subsubsection{Finance Industry}
Machine learning is growing in popularity in the finance industry. Banks are mainly using ML to find patterns inside the data but also to prevent fraud.
\subsubsection{Government organization}
The government makes use of ML to manage public safety and utilities. Take the example of China with the massive face recognition. The government uses Artificial intelligence to prevent jaywalker.
\subsubsection{Healthcare industry}
Healthcare was one of the first industry to use machine learning with image detection.
\subsubsection{Marketing}
Broad use of AI is done in marketing thanks to abundant access to data. Before the age of mass data, researchers develop advanced mathematical tools like Bayesian analysis to estimate the value of a customer. With the boom of data, marketing department relies on AI to optimize the customer relationship and marketing campaign.
\subsubsection{Supply Chain}
Machine learning gives terrific results for visual pattern recognition, opening up many potential applications in physical inspection and maintenance across the entire supply chain network.

Unsupervised learning can quickly search for comparable patterns in the diverse dataset. In turn, the machine can perform quality inspection throughout the logistics hub, shipment with damage and wear.

For instance, IBM's Watson platform can determine shipping container damage. Watson combines visual and systems-based data to track, report and make recommendations in real-time.

In past year stock manager relies extensively on the primary method to evaluate and forecast the inventory. When combining big data and machine learning, better forecasting techniques have been implemented (an improvement of 20 to 30 \% over traditional forecasting tools). In term of sales, it means an increase of 2 to 3 \% due to the potential reduction in inventory costs.

\paragraph{Google Car}
For example, everybody knows the Google car \textbf{Waymo}. The car is full of lasers on the roof which are telling it where it is regarding the surrounding area. It has radar in the front, which is informing the car of the speed and motion of all the cars around it. It uses all of that data to figure out not only how to drive the car but also to figure out and predict what potential drivers around the car are going to do. What's impressive is that the car is processing almost a gigabyte a second of data.


For our case we use logistic regression (discussed in Appendix app:4) and the algorithm is discussed in Proposal chapter \ref{ch:3}.
